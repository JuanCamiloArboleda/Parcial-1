\documentclass{article}
\usepackage[utf8]{inputenc}
\usepackage[spanish]{babel}
\usepackage{listings}
\usepackage{graphicx}
\graphicspath{ {images/} }
\usepackage{cite}

\begin{document}

\begin{titlepage}
    \begin{center}
        \vspace*{1cm}
            
        \Huge
        \textbf{Parcial 1}
            
        \vspace{0.5cm}
        \LARGE
        Arduino
            
        \vspace{1.5cm}
            
        \textbf{Dylan Saldarriaga}
         \textbf{Juan Camilo Arboleda}
         \textbf{Santiago Pereira Ramirez}
       
        \vfill
            
        \vspace{0.8cm}
            
        \Large
        Despartamento de Ingeniería Electrónica y Telecomunicaciones\\
        Universidad de Antioquia\\
        Medellín\\
        Marzo de 2021
            
    \end{center}
\end{titlepage}

\tableofcontents
\newpage
\section{Sección introductoria}\label{intro}
en este parcial podremos demostra de las aplicaciones con los integrados, microcontralador y tambien la programacion en c++.

\section{Analisis y Propuestas} \label{contenido}

Se debera de utilizar el integrado 74hc595 y un arduino, se debera de llegar a la forma de implentar una matriz de led 8x8 con los materiales anteriormente propuestos. 
para el planteamiento del problema primero:

-- idear una manera de conectar los 64 LEDS y que estos tenga un orden

--Realizar las diferentes funciones para la buena manipulacion del integrado.

-- realizar el algoritmo que nos permita relizar de una manera eficas patrones, los cuales seran introducidos por el usuario.




\end{document}
