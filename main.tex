\documentclass{article}
\usepackage[utf8]{inputenc}
\usepackage[spanish]{babel}
\usepackage{listings}
\usepackage{graphicx}
\graphicspath{ {images/} }
\usepackage{cite}

\begin{document}

\begin{titlepage}
    \begin{center}
        \vspace*{1cm}
            
        \Huge
        \textbf{Parcial 1}
            
        \vspace{0.5cm}
        \LARGE
        Arduino
            
        \vspace{1.5cm}
            
        \textbf{Dylan Saldarriaga}
        
         \textbf{Juan Camilo Arboleda}
         
         \textbf{Santiago Pereira Ramirez}
       
        \vfill
            
        \vspace{0.8cm}
            
        \Large
        Despartamento de Ingeniería Electrónica y Telecomunicaciones\\
        Universidad de Antioquia\\
        Medellín\\
        Marzo de 2021
            
    \end{center}
\end{titlepage}

\tableofcontents
\newpage
\section{Planteamiento del problema}\label{intro}
en este parcial podremos demostrar las aplicaciones con los integrados, microcontrolador y también la programación en c++ y arduino.\\

Como inicio del planteamiento aprendimos un poco mas sobre como usar integrado 74hc595n con el fin de iniciar su implementación, dandonos cuenta que este tiene la opción de usarse en modo "cascada".\\

Al implementar el modo cascada logramos usar los 64 leds con 8 integrados y 3 salidas lógicas del arduino; así dimos inicio a la parte de programación donde buscamos en primero lugar el funcionamiento de los 64 leds\\

Por medio de ciclos for recorremos la matriz 8x8, y dependiendo si hay un 0 o 1 en la matriz, el led se mantiene apagado o se enciende respectivamente.

Y le mando un pulso al reloj de registro de salida, para que almacene el tiempo entre cada iteraccion, dando inicio a la implementacion de las diferentes funciones con el fin lograr los objetivos propuestos.

\section{Analisis y Propuestas} \label{contenido}

Se debera de utilizar el integrado 74hc595 y un arduino, se debera de llegar a la forma de implentar una matriz de led 8x8 con los materiales anteriormente propuestos. 
para el planteamiento del problema primero:  \\

-- Idear una manera de conectar los 64 LEDS y que estos tenga un orden matricial\\

-- Realizar las diferentes funciones para la buena manipulacion de la matriz.\\

-- Diseñar el algoritmo que nos permita hacer uso de una manera eficas los patrones, los cuales seran introducidos por el usuario e impresos por los leds. \\

-- Implementación de funciones que permitan el uso de una o mas columnas, igualmente con las diagonales principales o filas\\

-- Manipulacion de arreglos, matrices y punteros para la indexación de datos y la implementacion de los algoritmos. \\


\section{Funciones Y Algoritmos Implementados.} \label{contenido}

//funcion que muestra la matriz\\
--  void MostrarMatriz(int **m,int nfilas,int ncolumnas);\\
función que no retorna nada, recibe como datos el espacio de memoria dentro de la matriz y se encarga de imprimir ordenadamente los valores de la matriz. \\

//funcion ingresar valores a la matriz \\
--  void CambiarNumMatriz(int m,int nfilas, int ncolumnas); \\
funcion que no retorna nada, recibe como parametros la matriz, el numero de fila y la columna a modificar y simplemente se encarga de modificar ese dato de 0 a 1. \\

//funcion que limpia la matriz \\
-- void LimpiarMatriz(int m,int nfilas,int ncolumnas); \\
funcion la cual no retorna nada, recibe la cantidad de filas y columnas de la funcion y se encarga de devolver esta a su estado incial. (llenda de 0). \\

//la funcion que permitira que la matriz de led se vizualice \\
-- void imagen(int m,int ma[]); \\
funcion que no retorna nada, recibe una matriz y un arreglo, y se encarga de mostrar en la matriz de leds el patron ingresado en el serial del arduino. \\

//funcion para segmentacion de columnas \\
-- void ParteCol(int m,int col1,int col2, int col); \\
Funcion que no retorna nada y es la encargada de segmentar una columna deseada, iniciando desde el led ingresado y finalizando en el led final ingresado por el usuario. \\

//funcion segmentacion de filas \\
 -- void ParteFila(int m,int col1,int col2, int col); \\
funcion encargada de segmentar la fila deseada, desde el led inicial ingresado hasta el led final ingresado por el usuario, recibe una matriz, columda deseada a segmentar y desde que led a que led desea segmentar. \\

//funcion segmentacion de diagonales \\
void segmentacionDiagonales(int m,int fila1, int columna1, int fila2, int columna2); \\
funcion encargada de segmentar una diagonal ingresada desde fila1,columna1 hasta fila2,columna2, recibe una matriz, y las coordenadas de ambos leds. \\


//funcion para encender todos los leds de la matriz \\
-- void Verificacion(int ma[]) \\
funcion encargada de comprobar el funcionamiento de todos los leds, encendiendolos. Esta funcion no retorna nada y recibe una matriz. \\


\section{Problemas de desarrollo que presentó}\label{contenido}

En el transcurso del problema se evidenciaron algunos inconvenientes uno de estos fue

-- Las conexiones de los leds y el uso del integrado 

-- Entender el orden de los leds de tal manera que al realizar el algoritmo funcionen de forma correcta.

-- Hallar la forma con la cual pudiesemos modificar la matriz para mostrar un patron deseado el cual será ingresado por el usuario.

-- La manipulacion del puerto serial del arduino. El arduino.

-- Trasladar la sintaxis del Qt al arduino de tal manera que el codigo quedase funcional y reconociese los valores ingresados.

-- Implementacion de los apuntadores y la memoria dinamica en el codigo del problema.

--La matriz (8n)x(8) 

\section{Evolución del algoritmo y consideraciones a tener en cuenta en la implementación}\label{contenido}

for(int i = 7;i>=0;i--) \\
  {
    for(int j = 7;j>=0;j--) \\
    {
      digitalWrite(ma[0], *(*(m+i)+j)); //buscamos en la matriz ma el valor el cual le pasaremos al serial.
donde i es igual a las filas y j las columnas.\\
      //se le da el valor para al serial el cual pasara el numero \\
  
      digitalWrite(ma[2], 0); \\
      digitalWrite(ma[2], 1);\\
      digitalWrite(ma[2], 0);\\

      digitalWrite(ma[1], 0);\\
      digitalWrite(ma[1], 1);\\
      digitalWrite(ma[1], 0);  \\    
    }    \\
  }   \\
\end{document}