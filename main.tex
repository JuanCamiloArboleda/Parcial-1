\documentclass{article}
\usepackage[utf8]{inputenc}
\usepackage[spanish]{babel}
\usepackage{listings}
\usepackage{graphicx}
\graphicspath{ {images/} }
\usepackage{cite}

\begin{document}

\begin{titlepage}
    \begin{center}
        \vspace*{1cm}
            
        \Huge
        \textbf{Parcial 1}
            
        \vspace{0.5cm}
        \LARGE
        Arduino
            
        \vspace{1.5cm}
            
        \textbf{Dylan Saldarriaga}
        
         \textbf{Juan Camilo Arboleda}
         
         \textbf{Santiago Pereira Ramirez}
       
        \vfill
            
        \vspace{0.8cm}
            
        \Large
        Despartamento de Ingeniería Electrónica y Telecomunicaciones\\
        Universidad de Antioquia\\
        Medellín\\
        Marzo de 2021
            
    \end{center}
\end{titlepage}

\tableofcontents
\newpage
\section{Planteamiento del problema}\label{intro}
en este parcial podremos demostrar las aplicaciones con los integrados, microcontrolador y también la programación en c++ y arduino.

\section{Analisis y Propuestas} \label{contenido}

Se debera de utilizar el integrado 74hc595 y un arduino, se debera de llegar a la forma de implentar una matriz de led 8x8 con los materiales anteriormente propuestos. 
para el planteamiento del problema primero:

-- Idear una manera de conectar los 64 LEDS y que estos tenga un orden matricial

-- Realizar las diferentes funciones para la buena manipulacion de la matriz.

-- Diseñar el algoritmo que nos permita hacer uso de una manera eficas los patrones, los cuales seran introducidos por el usuario e impresos por los leds.

-- Implementación de funciones que permitan el uso de una o mas columnas, igualmente con las diagonales principales o filas


\section{Algoritmo implementado} \label{contenido}

const int SER = 2;	//ENTRADA SERIAL

const int RCLK = 4;	//RELOJ REGISTRO DESPLAZAMIENTO

const int SRCLK = 5;  //REGISTRO DE SALIDA

int matriz[8][8]={{"0,0,0,0,0,0,0,0"},{"0,0,0,0,0,0,0,0"},{"0,0,0,0,0,0,0,0"},{"0,0,0,0,0,0,0,0"},{"0,0,0,0,0,0,0,0"},{"0,0,0,0,0,0,0,0"},{0,0,0,0,0,0,0,0"},{"0,0,0,0,0,0,0,0"}};


//funcion que recibe 4 variables tipo int, la primera el valor que se le dara a cada led y las siguientes los puertos analogos del arduino

void Verificacion(int numero, int serial ,int reloj,int registro);

//funcion que muestra la matriz

void MostrarMatriz(int m[8][8],int nfilas,int ncolumnas);

void letraA(int m[8][8]);

void recorrer(int m[8][8]);

void setup()
{
  Serial.begin(9600);
  
  //configuracion de puertos digitales de forma: OUTPUT
  
  pinMode(SER, OUTPUT);
  
  pinMode(RCLK, OUTPUT);
  
  pinMode(SRCLK, OUTPUT);
  
  //inicializar la entrada serial y los relojes en bajo
  
  digitalWrite(SER, 0);
  
  digitalWrite(RCLK, 0);
  
  digitalWrite(SRCLK, 0);
  
  /*
  for(int i = 1; i<=64; i++)
  
  {
  
    Verificacion(i,SER,RCLK,SRCLK);   
    
  } 
  */
  
  MostrarMatriz(matriz,8,8);
  
  letraA(matriz);
  
  Serial.println();
  
  MostrarMatriz(matriz,8,8);
  
  recorrer(matriz);
  
  
}

void loop()
{   
  
}

void Verificacion(int numero, int serial ,int reloj,int registro)

{
  digitalWrite(serial, numero);//se le da el valor para al serial el cual pasara el numero 
  
  
  digitalWrite(registro, 0);//falta pro explicar
  
  digitalWrite(registro, 1);
  
  digitalWrite(registro, 0);
  
  digitalWrite(reloj, 0);//falta por explicar
  
  digitalWrite(reloj, 1);
  
  digitalWrite(reloj, 0);  
  
  //delay(40);
  
}

void MostrarMatriz(int m[8][8],int nfilas,int ncolumnas)
{

    Serial.print("Imprimiendo matriz inicial: ");Serial.println();
    
    for (int x=0;x<nfilas;x++)
    
    {
    
        for (int y=0;y<ncolumnas;y++)
        
        {
        
            Serial.print(m[x][y]);
        }
        
        Serial.println();
        
    }
    
}


void recorrer(int m[8][8])

{

  for(int i = 7;i>=0;i-- )
  
  {
  
    for(int j = 7;j>=0;j-- )
    
    {
    
      digitalWrite(SER, m[i][j]);//se le da el valor para al serial el cual pasara el numero 
  
      digitalWrite(SRCLK, 0);//falta pro explicar
      
      digitalWrite(SRCLK, 1);
      
      digitalWrite(SRCLK, 0);
      
      digitalWrite(RCLK, 0);//falta por explicar
      
      digitalWrite(RCLK, 1);
      
      digitalWrite(RCLK, 0);
      
    }
    
  }   
  
}

void letraA(int m[8][8])

{
    for (int x=0;x<64;x++)
    
    {
    
        if (x<18)
        
        {
        
            m[0][x]=1;
            
        }
        
      	else if(x>21 && x<=25)
      	
        {
        
                m[0][x]=1;
                
        }
        
      	else if(x>29 && x<=41)
      	
        {
        
            m[0][x]=1;
            
    	}
    	
      	else if(x>45 && x<=49)
      	
        {
        
            m[0][x]=1;
            
    	}
    	
      	else if(x>53 && x<=57)
      	
        {
        
            m[0][x]=1;
            
    	}
    	
      	else if(x>61 && x<=64)
      	
        {
        
            m[0][x]=1;
            
    	}
    	
    }
    
}

\section{Problemas de desarrollo que presentó}\label{contenido}

En el transcurso del problema se evidenciaron algunos inconvenientes uno de estos fue

-- Las conecciones de los leds y el uso del integrado 

-- Entender el orden de los leds de tal manera que al realizar el algoridmo se entieran entre si

-- La forma de usar las funciones para imprimir un simbolo

-- Implementacion de los apuntadores y la memoria dinamica en el codigo del problema

\section{Evolución del algoritmo y consideraciones a tener en cuenta en la implementación}\label{contenido}


\end{document}
